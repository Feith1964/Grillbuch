%!TEX root = Vorlage_Buch.tex
\chapter{Gemüse und Pilze die unterschätzten Darsteller in der Küche}\label{Chapter3}

\lettrine[lines=3]{G}{emüse mach Spass, wetten dass?} In diesem Kapitel möchte ich euch zeigen, dass Gemüse zu Schade ist um als Beilage oder zu Kaninchenfutter degradiert wird sondern richtiges Soulfood, nicht nur für Veggies, ist. Es ist bunt, es ist vielseitig, es ist knackig (wenn es nicht getötet wird) und es ist vor allem vielseitig. Lassen wir uns auf das Experiment Gemüse ein.  In diesem Kapitel werden auch Pilze betrachtet, sie gehören allerdings weder zur Flora noch zur Fauna, sondern bilden ihr eigenes Reich. Ich habe die Pilze trotzdem hier aufgenommen, das ein eigenes Kapitel für diese Buch nicht gewinnbringend wäre.

\section{Gemüse als Beilage}

\subsection{Weißer Spargel, gedämpft}

\paragraph{Geräte}

\begin{itemize}[noitemsep]
	\item Holzkohlegrill
	\item Gasgrill
	\item Kamado Grill
\end{itemize}

\paragraph{Zutaten}

\begin{itemize}[noitemsep]
	\item 1,5 kg Spargel, geschält
	\item Saft einer Zitrone
	\item Olivenöl, genau soviel wie Zitronensaft
	\item 1 gute Prise Salz
	\item 1 gute Prise Zucker
	\item 3 Bögen Backpapier
	\item Aluminiumfolie
\end{itemize}
	
\paragraph{Zubereitung}
Den Zitronensaft, das Öl, das Salz und den Zucker mit einem kleinen Schneebesen 
verrühren. Je ein Drittel der Spargel auf einen Bogen Backpapier nebeneinander 
legen und mit einem Drittel der Marinade beträufeln. Das Backpapier 
einschlagen, sodass dichte Päckchen entstehen. Diese Päckchen in Alufolie 
einschlagen und bei ca. 200°C 20-25 Minuten auf der indirekten Zone bissfest 
garen. Die Spargel mit Lavendelbutter Kapitel~\ref{Chapter6} \vref{LavButter} 
bestreichen und servieren. Die Aromatik ist unbeschreiblich.

\subsection{Gegrillte Blumenkohlschnitzel}

\paragraph{Geräte}

\begin{itemize}[noitemsep]
	\item Gasgrill mit Plancha
\end{itemize}

\paragraph{Zutaten}

\begin{itemize}[noitemsep]
	\item 1 großer Blumenkohl
	\item grobes Himalaya-Salz aus der Mühle
	\item Olivenöl
\end{itemize}

\paragraph{Zubereitung}

Den Grill auf ca. 200°C vorheizen. Den Blumenkohl in ca. 2 cm dicke 
Scheiben schneiden. Die Blumenkohlscheiben mit Olivenöl beträufeln und 
salzen. Die Blumenkohlschnitzel auf der Plancha von beiden Seiten anbraten, 
bis sie stark gebräunt sind. Warm servieren und eventuell eine Vinaigrette 
dazu reichen.

\subsection{Gegrillter Rosenkohl}

\paragraph{Geräte}

\paragraph{Zutaten}

\paragraph{Zubereitung}

\subsection{Spitzkohl vom Grill}

\paragraph{Geräte}

\paragraph{Zutaten}

\paragraph{Zubereitung}

\paragraph{Wurzelgemüse aus dem Grillkorb}

\paragraph{Geräte}

\paragraph{Zutaten}

\paragraph{Zubereitung}

\subsection{Mediterranes Gemüse vom Grill}

\paragraph{Geräte}

\paragraph{Zutaten}

\paragraph{Zubereitung}



\section{Pilze und was sich daraus machen lässt}

\subsection{Steinpilz-Burger mit Kartoffelstroh}

\paragraph{Geräte}

\begin{itemize}[noitemsep]
	\item Gasgrill mit Plancha
	\item Fritteuse oder Airfryer
	\item Spiralschneider
\end{itemize}

\paragraph{Zutaten}

\begin{itemize}[noitemsep]
	\item 200 g Steinpilze
	\item 200 g Zwiebeln
	\item 1 Chinesischer Knoblauch
	\item Salz \& Pfeffer
	\item 1 Schuss süße Sahne
	\item 1 Kartoffel
	\item 1 Brioche-Bun
	\item Salat
	\item Orangen-Mayonnaise
\end{itemize}

\paragraph{Zubereitung}



