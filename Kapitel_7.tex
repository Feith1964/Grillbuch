%!TEX root = Vorlage_Buch.tex
\chapter{Gewürze, Saucen und Dips}\label{Chapter7}
\lettrine[lines=3]{G}{ewürze} und Kräuter sind aus der Küche nicht 
wegzudenken. 
Gewürze und Kräuter werden eingesetzt um die Geschmacksvielfalt 
verschiedener Regionen
und deren Fusionen widerzuspiegeln.
Sie wirken konservierend, sind gesundheitsfördernd und steigern das 
Wohlbefinden.
Sie enthalten Säuren, Zucker, ätherische Öl und andere Komponenten die 
einzigartig in
Geschmack und Wirkung sind. Das bilden verschiedener Komplexe durch 
chemische 
Reaktionen mit dem Grillgut führen Veränderungen der Textur, entwickeln 
Aromen und konservieren.
   

\section{Gewürze und Gewürzmischungen}
Die Mischungen die aus verschiedenen Gewürzen zusammengesetzt sind, sind 
zum Teil bestens gehütete Geheimnisse. Gewürzmischungen  sind unter anderem 
Alleinstellungsmerkmale
in der professionellen Küche und dem Gewürzhandel. Vor allen im Orient findet 
man eine schier
unendliche Zahl vom Familienrezepten verschiedener Gewürzmischungen. In 
der kreolischen Küche  stellt sich nicht anders dar.
In der deutschen Küche haben die Gewürze ebenfalls Einzug gehalten, das ist 
vor allem in der Grillszene und der Gastronomie zu verdanken.

\subsection{Kansas City Rib Rub}\label{Kansas}
Kansas City ist die Welthauptstadt des Barbecue, in der die jährliche 
Grillweltmeisterschaft, der größte Grillwettbewerb  der Welt,  veranstaltet wird. 
Es ist daher nicht verwunderlich, dass es auch Rubs gibt die nach Kansas City 
benannt wurden. Hier ist ein Rub der perfekt zu Schweinefleisch passt, da in 
Kansas City vor allen Rippchen serviert werden.
\paragraph{Für den Rub}

\begin{itemize}[noitemsep]
	\item 8 EL brauner Zucker
	\item 4 EL Paprikapulver
	\item 1 EL Salz
	\item 1 EL schwarzer Pfeffer, ganze Körner
	\item 1 EL Zwiebelpulver
	\item 1 EL Knoblauchpulver
	\item 1 TL Cayennepfeffer
	\item 1 TL Chilipulver
\end{itemize}

\paragraph{Zubereitung}

Um das Aroma dieses Rubs zu vertiefen röste ich den Schwarzen Pfeffer in der 
Pfanne an, bis 
knackende Geräusche entstehen. Nach dem Rösten wird der Pfeffer im Mörser 
zerkleinert. Die Zutaten mischen, fertig.

\subsection{Magic Dust}\label{Magic}

">Magic Dust"< ist ein vielseitiger Rub  und aus der Grillszene nicht weg zu 
denken. Durch die Kombination aus rauchigen, würzigen und süßen Noten ist 
er universell einsetzbar und passt zu fast allem das man grillen kann. Diese 
Mischung bringt durch ihre Kombination der Aromen und würzigen Noten den 
perfekten Geschmack auf Fleisch und andere Speisen. Typische Zutaten sind 
Paprika, brauner Zucker, Salz, Knoblauch- und Zwiebelpulver, Senfpulver, 
Chilipulver und manchmal ein Hauch Cayennepfeffer für zusätzliche Schärfe.

\paragraph{Für den Rub}

\begin{itemize}[noitemsep]
	\item 3 EL Paprikapulver, edelsüß
	\item 2 EL grobes Meersalz
	\item 2 EL brauner Zucker
	\item 2 EL Chilipulver Zucker
	\item 1 EL geräuchertes Paprikapulver
	\item 1 EL Senfpulver
	\item 1 EL Knoblauchpulver
	\item 1 EL Zwiebelpulver
	\item 1 TL Cyannepfeffer
\end{itemize}

\paragraph{Zubereitung}

Um das Aroma dieses Rubs zu vertiefen röste ich den Schwarzen Pfeffer in der 
Pfanne an, bis 
knackende Geräusche entstehen. Nach dem Rösten wird der Pfeffer im Mörser 
zerkleinert. Die Zutaten mischen, fertig.

\subsection{Chicken Rub}\label{Chicken}
Das Chicken Rub ist das perfekte Rub für Hähnchen. Ob Drumsticks, komplette 
Keulen oder das Road Killed Chicken. Mit diesem leicht herzustellenden Rub 
haben wir ein universelles Hähnchengewürz mit Geschmacksgarantie.

\begin{itemize}[noitemsep]
	\item 5 EL Paprikapulver, edelsüß
	\item 3 EL Salz
	\item 2 EL schwarzer Pfeffer, frisch gemahlen
	\item 2 EL Zwiebelpulver
	\item 2 EL Knoblauchpulver
	\item 1 EL Oregano, getrocknet
	\item 1 EL Thymian, getrocknet
	\item 1 EL Cayennepfeffer
\end{itemize}

\paragraph{Zubereitung}

Die Zutaten mischen, fertig.

\subsection{Bratkartoffelgewürz}

\paragraph{Für das Bratkartoffelgewürz}\label{Bratkartoffelgewürz}

\begin{itemize}[noitemsep]
	\item 50 g Meersalz
	\item 20 g Bockshornklee
	\item 1/2 TL Koriandersaat, gemahlen
	\item 1/2 TL  Knoblauchpulver
	\item 1/2 TL  Paprika, edelsüß
	\item 1/2 TL Zwiebelpulver
	\item 1 Prise geräucherter Paprika
	\item 1 Prise Muskatnuss, frisch gerieben
	\item 1 Prise Fenchelsaat, gemahlen
	\item 1 Prise Cumin, gemahlen
	\item 1/2 TL Kaschmir-Chili-Pulver
	\item 1 Prise Schwarzer Pfeffer, gemahlen
	\item 1 kleine Prise Nelken, gemahlen
\end{itemize}

\paragraph{Zubereitung}

Die Zutaten mischen, fertig.

\subsection{Gewürzmischung für das Schwarze Chili}\label{Gewürzmischung}

\paragraph{Zutaten}

\begin{itemize}[noitemsep]
	\item 1 EL gekörnte Brühe
	\item 3 TL Chilipulver (aus eigener Herstellung, das hat richtig Dampf)
	\item 20 g Zartbitterschokolade (hier 70\% Kakaoanteil)
	\item 1 TL Kümmelpulver
	\item 1 gestr. TL Oregano
	\item 1 gestr. TL Koriander
	\item 1 gestr. TL Estragon
	\item 1/2 TL Kleiner Langpfeffer (ist der kleine Bruder des Bengal-Pfeffers, Bezugsquelle \url{www.gewuerzshop-mayer.de})
	\item 1/2 TL Zimt (gibt dem ganzen eine festliche Note)
	\item 2 EL Ras el Hanut ("`Chef der Küche"' der macht die ganze Sache Rund)
\end{itemize}

\paragraph{Zubereitung}

Die Zartbitterschokolade in eine Schüssle reiben, die restlichen Zutaten 
hinzufügen, gut durchmischen, fertig. Soll das Gewürz in größeren Mengen auf 
Vorrat hergestellt werden, kann die Zartbitterschokolade durch Kakaopulver 
ersetzt werden. In diesem Falle nur 2/3 des  Gewichtanteils verwenden.

\section{Öle und Soßen}
{Vom Arganöl bis zur Zitronenmayo}

\subsection{Chiliöl-mit-Crunch}
{Knuspriges Chili-Öl}
hört widersprüchlich an,  es gibt aber tatsächlich ein Rezept für Chili-Öl mit 
knusprigen Zutaten. Das für den universellen Einsatz wie geschaffen ist. Ich 
empfehle je nach Leidensfähigkeit das Öl in verschiedenen Schärfestufen 
herzustellen. Um ein schokoladiges Dessert zu verfeinern ist es nicht förderlich 
ein höllisch scharfes Chili-Öl zu verwenden. Während zum verfeinern einer 
Pizza oder Spareribs schon ein Öl mit etwas mehr Dampf benutzt werden kann.

Das Grundrezept kommt mit wenigen Zutaten aus und liefert bereits ein 
Hammerergebnis.
Ich stelle euch außerdem einige Ideen vor um das Rezept für den eigenen 
Geschmack anzupassen.
\newline

\paragraph{Für das Grundrezepts:}

\begin{itemize}[noitemsep]
	\item 1 Liter neutrales Speiseöl, etwa Raps
	\item 90 g getrocknete Chilis oder Chiliflocken,  je nach Vorliebe auch mehr
	\item 5-8 Schalotten
	\item 1/2 Knolle Knoblauch
	\item 50 ml Sojasoße
	\item 20 g Zucker (ich nehme gerne Rohrzucker, der süßt nicht so stark wie 
	Rübenzucker, das kann nach belieben ausgeglichen werden) 
\end{itemize}

\paragraph{Optional:}

\begin{itemize}[noitemsep]
	\item 50 ml Balsamico Essig
	\item Gewürze wie Zimt,  Ingwer,  Anis
	\item 100 g Sesam oder gehackte Erdnüsse/ Cashewkerne
\end{itemize}

\paragraph{Zubereitung}

Das Öl in einem großen Kochtopf auf mittlere Hitze bringen und die Gewürze 
wie Zimt,  Anis und Ingwer darin für etwa 10 Minuten frittieren, um das Öl zu 
aromatisieren. Die Gewürze anschließend mit einer Schöpfkelle aus dem Öl 
fischen und entsorgen.
Die Schalotten und den Knoblauch schälen und in möglichst feine Scheiben 
schneiden. Wer einen feinen Hobel oder eine Mandoline hat, nimmt sie zu Hilfe. 

Die Schalotten brauchen deutlich länger als der Knoblauch. Schalotten sind 
nach 10 bis 15 Minuten fertig, Knoblauch schon nach ein bis zwei Minuten. Gibt 
man beide gleichzeitig rein, verbrennt Letzterer nur. Daher: Schalotten und 
Knoblauch nacheinander im aromatisierten Öl knusprig braun frittieren lassen. 
Wenn die Blubberblasen deutlich nachlassen, ist die meiste Flüssigkeit aus 
dem Gemüse verschwunden, und wir haben ein Maximum an Crunch erreicht – 
danach kommt nur noch Asche. Also fix mit der Schöpfkelle herausnehmen und 
beiseite legen.
Die Chiliflocken und das knusprige Element wie Sesam oder Nüsse in eine 
hitzefeste Schale oder einen Topf legen, dann das heiße Öl obendrauf gießen. 
Niemals direkt in ein Glas gießen, dieses könnte springen! Das Öl ist noch lange 
heißer, als es aussieht.
Alles gut abkühlen lassen (etwa 30 bis 45 Minuten) und dann mit Zucker, 
Sojasoße und Essig abschmecken. Vorsicht: Wenn das Öl noch zu heiß ist, 
blubbert es bei der Zugabe von Essig und anderen Flüssigkeiten. Also lieber 
warten.
Die knusprigen Schalotten und den Knoblauch unterrühren und alles mit einer 
Kelle in sterile Gläser umfüllen, etwa abgekochte Marmeladengläser.
\paragraph{Variationen} Wer sich austoben möchte, kann gerade bei diesem 
Rezept sehr viel ausprobieren. Ein paar Ideen:

Zunächst gibt es eine riesige Varianz an Chilisorten, von mittel- und 
südamerikanischen hin zu asiatischen. Chipotles (geräucherte Jalapeños) 
bringen etwa einen tollen Rauchgeschmack mit rein. Thailändische Bird’s Eye 
Chilis sind scharf wie die Hölle, koreanisches Gochugaru hingegen oft 
vergleichsweise mild. Auch frische Chilis können verwendet werden, dann sinkt 
aber die Haltbarkeit.
Für mehr Tiefe und Herzhaftigkeit kann man auch getrocknetes Pilzpulver 
hinzugeben oder etwas Würzsoße mit Glutamat.

Wer etwa Fünf-Gewürze-Pulver oder weißen Pfeffer hinzufügt, macht dies am 
besten erst, wenn das Öl weit genug abgekühlt ist, um in Gläser umgefüllt zu 
werden. Dasselbe gilt für andere Pulvergewürze.
Szechuanpfeffer, Curryblätter oder Kaffirlimettenblätter können dem Öl weitere 
leckere Aromen hinzufügen.
Wer mag, kann das Öl oder Anteile davon ersetzen. Sesamöl etwa hat ein tolles 
Aroma, sollte jedoch nicht zu stark erhitzt werden (sonst geht ebenjenes 
Aroma flöten). Sesamöl also erst zum Verfeinern am Ende des Prozesses mit 
ins Glas geben. Soja- oder Erdnussöl bringen viel Eigengeschmack mit und 
können von Anfang an verwendet werden – kosten aber auch mehr als Rapsöl.
Man kann die Menge der knusprigen Einlage deutlich nach oben skalieren. 
Gerade geröstete und gehackte Erdnüsse sind günstig zu bekommen.

\paragraph{Zutaten für die Orangen-Chili-Butter}

\begin{itemize}[noitemsep]
	\item 4 EL Geklärte Butter
	\item 1 Rote Chilischote (Cayenne)
	\item 1/2 Bio-Orange
	\item 1 TL Fenchelsamen
\end{itemize}

\paragraph{Zubereitung}

Die Butter erhitzen. Die Chilischote entkernen und in sehr feine Ringe schneiden 
und zusammen 
mit dem Fenchelsamen in der Butter bei 
schwacher Hitze 1 bis 2 Minuten garen. In der Zwischen Zeit mit einem 
Sparschäler von der 
gewaschenen Orange die Hälfte der Schale 
abschälen und in sehr feine Streifen schneiden. Die Orangenstreifen unter die 
Butter mischen 
und alles auf einem Reschaud warmhalten.

\subsection{Orangen-Chile-Butter}

Die Orangen-Chili-Butter ist eins schnelle Topping für Pasta, Baguette oder 
Geflügel. Das 
einfache Rezepte hat eine fantastische 
Aromatik, eine milde Säure und eine dezente Schärfe.

\paragraph{Zutaten für die Orangen-Chili-Butter}\label{OrangenChili}

\begin{itemize}[noitemsep]
	\item 4 EL Geklärte Butter
	\item 1 Rote Chilischote (Cayenne)
	\item 1/2 Bio-Orange
	\item 1 TL Fenchelsamen
\end{itemize}

\paragraph{Zubereitung}

Die Butter erhitzen. Die Chilischote entkernen und in sehr feine Ringe schneiden 
und zusammen 
mit dem Fenchelsamen in der Butter bei 
schwacher Hitze 1 bis 2 Minuten garen. In der Zwischen Zeit mit einem Sparschäler von der 
gewaschenen Orange die Hälfte der Schale 
abschälen und in sehr feine Streifen schneiden. Die Orangenstreifen unter die Butter mischen 
und alles auf einem Reschaud warmhalten.

\subsection{Ketchup}
Ketchup ist beim Grillen nicht wegzudenken. Pur oder als Basis für andere 
Saucen. Ein gutes Ketchup ist ein muss für jeden BBQ-Fan.
Ich habe hier ein Rezept zusammengestellt, das nicht nur super schmeckt 
sondern auch leicht nachzukochen ist. 
\newline

\paragraph{Grundrezept für das Ketchup}\label{Ketchup}

\begin{itemize}[noitemsep]
	\item 1,5 kg Tomaten (am Besten Flaschentomaten)
	\item 1 Zwiebel
	\item 1 Zehe Knoblauch
	\item 100 g Brauner Zucker
	\item 80 ml Essig (mild, z.B. Rotweinessig oder Apfelessig)
	\item 2 Gewürznelken
	\item 2 Lorbeerblätter
	\item 1/4 TL Five Spices
	\item TL Salz
	\item 1 EL Öl
\end{itemize}

\paragraph{Zubereitung}

Zwiebel und Knoblauch hacken. Die Tomaten waschen, häuten und in Stücke 
schneiden.

Öl in einem Topf erhitzen, Zwiebeln und Knoblauch darin langsam dünsten, bis 
sie glasig sind.
Dann die kleingeschnittenen Tomaten und die Gewürze (außer Essig und 
Zucker) zugeben. Unter gelegentlichem
Rühren ca. 30 Minuten köcheln lassen.

Jetzt Zucker und Essig zugeben, unterrühren, einmal aufkochen. Dann kannst 
Du den selbstgemachten 
Ketchup final abschmecken und noch warm in die Gläser abfüllen. Die Gläser 
dann sofort fest verschießen.

\subsection{Habanero BBQ Sauce}
Wer nicht gerne scharf isst, sollte diese Rezept auf keinem Fall ausprobieren, 
denn diese Sauce hat es in sich. Habaneros sind bekannt für ihre feurige 
Schärfe, aber auch für ihren fruchtigen Geschmack. Beides kommt in dieser 
Sauce perfekt zur Geltung.
\newline

\paragraph{Für die Habanero BBQ Sauce}

\begin{itemize}[noitemsep]
	\item 1 Tube Tomatenmark à 200 ml (dreifach konzentriert)
	\item 4 Habaneros, frisch
	\item 1 Zwiebel
	\item 120 ml Apfelessig
	\item 120 ml Wasser
	\item 30 g Senf (je nach Geschmack)
	\item 2 EL Sojasauce
	\item 1 TL Paprikapulver (geräuchert)
	\item Salz
	\item Pfeffer
\end{itemize}

\paragraph{Zubereitung}

Die Habanero BBQ Sauce ist schnell zubereitet und benötigt nicht viele Zutaten.
\textbf{Wichtig:} \emph{Ziehe beim Schneiden der Habaneros Handschuhe an, 
	denn brennen sollen die 
	scharfen Chilischoten nur beim Essen, nicht bei der Zubereitung.} Wenn du 
	die Habaneros etwas 
„entschärfen“ möchtest, entferne die Kerne teilweise oder komplett.

Sie sieht zwar nicht so aus, aber die Habanero BBQ Sauce ist unglaublich sc
harf und lecker zugleich.

Die Zwiebel fein würfeln. Habaneros in feine Stücke schneiden (Handschuhe 
anziehen).
Die Zwiebelwürfel in etwas Öl glasig dünsten. Dann die Habaneros und alle 
anderen Zutaten 
hinzugeben und verrühren.
Alles etwa eine halbe Stunde köcheln lassen.
Das Rezept sieht vier Habanero-Schoten vor, du kannst die Menge und damit 
den Schärfegrad aber 
ganz nach deinen Wünschen variieren.

\subsection{Klassische amerikanische BBQ-Sauce}
Diese Sauce sollte bei keinem BBQ fehlen. Der Klassiker unter den BBQ Saucen 
ist ein absoluter Allrounder und passt zu ziemlich jedem Grillgut.
Am besten schmeckt sie natürlich, wenn du die selber machst.

Tomaten sorgen für einen fruchtigen Geschmack, brauner Zucker für die Süße, 
Essig für die Säure und Chili für eine angenehme Schärfe. Das
sind die wichtigsten Grundzutaten für die klassisch amerikanische BBQ Sauce.
\newline

\paragraph{Für die BBQ Sauce}

\begin{itemize}[noitemsep]
	\item 200 ml Tomaten passiert
	\item 100 ml Ketchup (Rezept \vref{Ketchup})
	\item 50 ml Rotweinessig
	\item 50 g Brauner Zucker
	\item 1 Zwiebel
	\item 2 Zehen Knoblauch
	\item 50 ml Wasser
	\item 1 TL Paprikapulver
	\item 1/2 Chiliflocken
	\item 1/2 Senf
	\item 1/2 Salz
\end{itemize}

\paragraph{Zubereitung}

Knoblauch und Zwiebeln fein würfeln und glasig dünsten.

Dann alle weiteren Zutaten zugeben und etwa eine halbe Stunde köcheln 
lassen, bis die gewünschte Konsistenz erreicht ist. Fertig.

\subsection{Salsa de las albóndigas}\label{SalsaAlbondigas}

Die Salsa de las Albóndigas wurde von den Mauren im 13ten Jahrhundert, nach 
Spanien gebracht. Die Salsa ist ein Bestandteil des 
Gerichtes Albóndigas en salsa. Ich habe die Salsa in diesem Kapitel 
untergebracht, da diese Salsa auch zu anderen Gerichten gegessen 
wird.

\begin{itemize}[noitemsep]
	\item 1 mittlere Zwiebel
	\item 1 frische rote Paprika
	\item 1 Knoblauchzehe
	\item 1 EL Tomatenmark
	\item 1 TL Zucker
	\item 1 Glas Sherry fino (150 ml), alt. Weißwein 
	\item 1 TL Weinessig
	\item 3 – 4 frische Tomaten
	\item Tomaten Passata, 300 ml (bei Bedarf mehr)
	\item 1 TL Pimenton picante (Räucherpaprika oder Paprika edelsüß)
	\item Frischer Chili oder Cayennepfeffer (optional)
	\item 1 EL Petersilie oder Koriander zum Dekorieren
\end{itemize}

\paragraph{Zubereitung}

Zwiebel, Knoblauch und Paprikawürfel glasig andünsten. Pimenton, Zucker und 
Tomatenmark 
unterrühren, kurz anrösten und mit einem 
Schuss Sherry ablöschen. Danach die frischen Tomatenstücke und Tomaten 
Passata 
dazugeben. Mit dem restlichen Sherry auffüllen und 
mit Salz, Pfeffer, Chili abschmecken und etwa 20 Minuten bei niedriger 
Temperatur köcheln bis 
sie sämig ist. Die Salsa darf nicht 
sprudelnd kochen.

\subsection{Lavendelbutter}\label{LavButter}

\paragraph{Zutaten}

\begin{itemize}[noitemsep]
	\item 100 g Butter
	\item 2 Zweige Lavendel
	\item 1 gute Prise Salz
\end{itemize}

\paragraph{Zubereitung}

Die Butter eine Stunde vor Zubereitung aus dem Kühlschrank nehmen. Lavendel 
klein schneiden 
und mit dem Salz unter die Butter mischen. Die Butter zu einer Kugel formen 
und kühl stellen.


\subsection{Uwe's mediterrane Kräuterbutter}\label{Kräuterbutter}

\paragraph{Zutaten}

\begin{itemize}[noitemsep]
	\item 250 g Butter
	\item 1 EL frischer, gehackter Thymian 
	\item 1 EL frischer, gehackter Oregano
	\item 1 EL frisches, gehacktes Basilikum
	\item 1/4 getrocknete Tomate
	\item 1 TL Kaschmir-Chilipulver
	\item Salz
\end{itemize}

\paragraph{Zubereitung}

Die Butter eine Stunde vor der Zubereitung aus dem Kühlschrank nehmen. Die 
getrocknete Tomate in kleine Stücke schneiden und Zusammen mit den 
gehackten Kräutern und dem Kaschmir-Chilipulver zu Butter geben und das 
ganze gut mischen. Mit Salz abschmecken. Die Butter wieder kaltstellen. Nach 
ca. einer Stunde aus der Kühlung nehmen und je nach Bedarf, in Kugeln, 
Nocken, einer Rolle oder gar nicht formen. 
	
\subsection{Uwe's leckere Orangenmayonnaise}\label{Orangenmayo}

\paragraph{Geräte}

\begin{itemize}[noitemsep]
	\item Stabmixer
	\item Messbecher
\end{itemize}

\paragraph{Zutaten}

\begin{itemize}[noitemsep]
	\item 1 Ei (M)
	\item 1 TL Delikatesssenf
	\item 1 EL Orangensaft
	\item 1 TL Zitronensaft
	\item 200 ml Rapsöl (oder ein anderes geschmacksneutrales Öl)
	\item Salz \& Pfeffer
	\item Orangentrester
\end{itemize}

\paragraph{Zubereitung}

Alle Zutaten sollten Zimmertemperatur haben, nur so können sie sich perfekt 
verbinden. Das Eigelb, Senf, Orangen- und Zitronensaft vermengen. Öl unter 
Rühren zugeben, am Anfang sehr langsam, später dann schneller. Sobald die 
Masse die gewünschte Konsistenz erreicht hat, wird der Orangentrester 
zugeben und die Mayonnaise mit Salz und Pfeffer abgeschmeckt.

\subsection{Uwe's Kürbiskernpesto mit Chili}

\paragraph{Geräte}

\begin{itemize}[noitemsep]
	\item Standmixer
	\item Pfanne
\end{itemize}

\paragraph{Zutaten}

\begin{itemize}[noitemsep]
	\item 240 g Kürbiskerne
	\item 240 g Parmesan
	\item 6 kleine Chilis, rot
	\item 6 kleine Knoblauchzehen
	\item 6 Handvoll Basilikumblätter
	\item 480 ml Olivenöl
	\item 12 EL Kürbiskernöl
	\item 6 TL Orangentrester
	\item 12 TL Currypulver (Madras)
	\item 6 Messerspitzen Chili-Pulver
	\item Salz \& schwarzer Pfeffer
\end{itemize}

\paragraph{Zubereitung}

Die Kürbiskerne in einer Pfanne ohne Fett rösten und anschließend abkühlen 
lassen.

Den Parmesan fein reiben und die Chili sehr fein würfeln und zur Seite stellen.

Knoblauchzehe, Kürbiskerne, Orangenabrieb, Basilikum, Olivenöl, Curry und 
Chilipulver in den Mixer geben und fein pürieren. Anschließend den Parmesan 
zusammen mit den Chili-Würfeln und dem Kürbiskernöl unter die Masse 
rühren, nicht mehr mixen. Das ganze mit Salz und Pfeffer abschmecken.