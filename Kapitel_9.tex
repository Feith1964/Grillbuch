%!TEX root = Vorlage_Buch.tex
\chapter{Fleisch Cuts}\label{Chapter9}

\lettrine[lines=3]{F}{leisch} ist nicht nur ein Grundnahrungsmittel, sondern auch ein 
Eckpfeiler kulinarischer Kreativität. In diesem Kapitel tauchen wir tief in die 
faszinierende Welt der Fleischzuschnitte ein. Ob Rind, Schwein, Lamm oder 
Geflügel – jeder Zuschnitt hat seine eigene Geschichte und bietet einzigartige 
Geschmacks- und Texturerlebnisse.

Ein Verständnis für die verschiedenen Fleischzuschnitte ist nicht nur für 
Profiköche von Bedeutung, sondern auch für ambitionierte Hobbyköche. 
Wissen über Fleischzuschnitte bedeutet, die besten Teile für bestimmte 
Gerichte auszuwählen, optimale Garzeiten zu bestimmen und den vollen 
Geschmack aus jedem Bissen herauszuholen. Vom zarten Filet bis zum 
schmackhaften Bratenstück – jedes Stück Fleisch birgt eigene Geheimnisse 
und kulinarische Möglichkeiten.

In den folgenden Seiten werden wir die verschiedenen Fleischzuschnitte 
detailliert vorstellen. Sie lernen, wie man die besten Stücke auswählt, welche 
Zubereitungsmethoden sie erfordern und wie man sie perfekt in Szene setzt. 
Lassen Sie sich inspirieren, entdecken Sie neue Favoriten und verfeinern Sie 
Ihre Kochkünste.

Willkommen in der Welt der Fleischzuschnitte – wo Wissen auf Genuss trifft 
und jedes Gericht ein Meisterwerk wird!

\section{Amerikanische Fleischzuschnitte vom Rind}

Die amerikanische Methode, Rindfleisch zu zerlegen, unterscheidet sich von ~
der europäischen und bietet eine Vielzahl von Cuts siehe 
Abbildung~\vref{fig:Übersicht}, die sich durch Geschmack und 
Verwendungsmöglichkeiten auszeichnen. Hier sind einige der bekanntesten 
amerikanischen Rindfleischzuschnitte:

\begin{description}
	
	\item [Chuck (Schulter)] Geschmacksintensiver Cut aus dem Nacken- und 
	Schulterbereich. Enthält viel Bindegewebe und Fett. 
	Ideal für Schmorgerichte, Eintöpfe und Suppen.
	
	\item [Rib (Rippenstück)] Saftiges und gut marmoriertes Fleisch aus dem 
	Rippenbereich. Beinhaltet Ribeye Steaks und Prime Rib.
	Hervorragend zum Grillen, Braten oder für den Ofen.
	
	\item [Short Plate (Bauchlappen)] Entstammt dem unteren Brustkorb. Das 
	Fleisch ist grobfaserig und aromatisch.
	Ideal für kurze Rippchen, Flank Steak und Skirt Steak.
	
	\item [Brisket (Bruststück)] Grobfaseriges Stück aus der Brust. Erfordert 
	lange Garzeiten, um zart zu werden.
	Perfekt zum Räuchern, Schmoren oder für BBQ.
	
	\item [Fore Shank (Vorderer Unterschenkel)] Enthält viel Bindegewebe. 
	Wird oft für geschmortes Fleisch verwendet.
	Ideal für Schmorgerichte und Suppen.
	
	\item [Short Loin (kurzes Lendenstück)] Zartes Fleisch aus dem 
	Rückenbereich. Beinhaltet T-Bone Steak, Porterhouse und Strip Steak.
	Perfekt zum Grillen oder Braten.
	
	\item [Sirloin (Hüfte)] Kräftiges, mageres Fleisch aus dem Hüftbereich. 
	Beinhaltet Top Sirloin und Bottom Sirloin.
	Gut geeignet zum Grillen, Braten oder für Steaks.
	
	\item [Tenderloin (Filet)] Der zarteste Cut, entnommen aus dem inneren 
	Lendenbereich. Sehr mager.
	Ideal für edle Gerichte, oft in der Pfanne gebraten oder im Ofen gegart.
	
	\item [Round (Keule)] Mageres Fleisch aus der hinteren Keule. Enthält 
	verschiedene Steaks und Bratenstücke. 
	Hervorragend für Schmorgerichte, Braten und Steaks.
	
	\item [Flank (Bauch)] Dünner, langer Cut aus dem Bauchlappen. 
	Grobfaserig und aromatisch.
	Ideal für mariniertes und kurz gebratenes Fleisch, wie Flank Steak.
	
	\item [Skirt (Saumfleisch)] Ein Cut aus dem Zwerchfell, lang und flach mit 
	intensiven Geschmack.
	Perfekt für Fajitas, Stir-Fries und mariniertes Grillfleisch.
	
	\item [Tri-Tip (Bürgermeisterstück)] Dreieckiger Cut aus dem unteren 
	Sirloin. Kräftiger Geschmack und relativ mager.
	Gut für Grillen, Braten und Räuchern.
	
	\item [Hanging Tender (Nierenzapfen)] Der Hanging Tender, auch bekannt 
	als Hanger Steak oder Nierenzapfen, ist ein weniger bekannter, aber 
	unglaublich geschmacksintensiver Cut
	Gut zum grillen oder schnellen anbraten.
	
	\item [Der Flat Iron,] auch bekannt als Schaufelstück oder Butlers' 
	Steak, ist ein relativ neuer und sehr beliebter Cut aus der Schulter des 
	Rindes.
	Gut zum grillen oder schnellen anbraten.
\end{description}

Diese Vielfalt an Cuts bietet eine breite Palette an Möglichkeiten für 
verschiedene Gerichte und Kochmethoden,
um das Beste aus jedem Stück Rindfleisch herauszuholen. 

\begin{figure}[htbp]
	\centering
	\begin{minipage}{1\textwidth}
		\centering
		\includegraphics[width=.9\linewidth]{pics/Übersicht_amerikanische_Cuts}
		\captionof{figure}{Übersicht amerikanische Cuts vom Rind}
		\label{fig:Übersicht}
	\end{minipage}
\end{figure}
\newpage


\section{Amerikanische Fleischzuschnitte vom Schwein}

Hier ist eine umfassende Liste der gängigen amerikanischen siehe Abbildung~\vref{fig:Übersicht1} 
und Fleischzuschnitten spanischen siehe Abbildung~\vref{fig:Übersicht3} vom Schwein, 
zusammen mit einer kurzen Beschreibung ihrer Eigenschaften und Verwendungszwecke:

\begin{description}
	\item [Boston Butt  (Schulterstück)]
	Ein Cut aus dem oberen Schulterbereich. Es ist gut marmoriert und reich an 
	Bindegewebe.
	Ideal für Pulled Pork, Schmorgerichte und BBQ.
	
	\item [Picnic Shoulder (Schulterstück unten)] 
	Ein Cut aus dem unteren Schulterbereich. Es ist fettärmer als das Boston 
	Butt.
	Hervorragend für langsames Garen, Schmorgerichte und geräuchertes 
	Fleisch.
	
	\item [Loin (Lende)] 
	Ein mageres und zartes Stück Fleisch, das entlang der Wirbelsäule des 
	Schweins geschnitten wird.
	Perfekt für Koteletts, Braten und Medaillons.
	
	\item [Baby Back Ribs (Rückenspeckrippen): ] 
	Diese Rippchen stammen aus dem oberen Rückenbereich. Sie sind kürzer 
	und zarter als andere Rippchen.
	Ideal für Grillen oder BBQ.
	
	\item [Spare Ribs (Schälrippen)] 
	Größere und fleischigere Rippchen, die aus dem Bauchbereich stammen.
	Perfekt zum langsamen Garen und Räuchern.
	
	\item [St. Louis Style Ribs] 
	Diese Rippchen sind eine geschnittene Variante der Spare Ribs, wobei das 
	knorpelige Brustbein entfernt wurde.
	Ideal für Grillen und BBQ.
	
	\item [Pork Belly (Schweinebauch)] 
	Ein sehr fettes Stück Fleisch aus dem Bauchbereich des Schweins. Es hat 
	einen intensiven Geschmack.
	Hervorragend für knusprigen Schweinebauch, Bacon und 
	Schweinebauchscheiben.
	
	\item [Ham (Schinken)] 
	Ein Cut aus der hinteren Keule des Schweins. Kann geräuchert, gepökelt 
	oder frisch sein.
	Perfekt für Schinkenbraten, Schinkensteaks oder gepökelten Schinken.
	
	\item [Pork Hock (Eisbein/Kniekehle)] 
	Ein Cut aus dem unteren Teil des Beins. Es ist geschmacksintensiv und 
	enthält viel Bindegewebe.
	Ideal zum langsamen Garen und Schmorgerichte.
	
	\item [Pork Jowl (Schweinebacke)] 
	Ein Cut aus dem Wangenbereich des Schweins. Es ist fettig und 
	geschmacksintensiv.
	Häufig für Guanciale, geräuchertes Fleisch und Schmorgerichte.
	
	\item [Tenderloin (Filet)] 
	Der zarteste Cut des Schweins, entnommen aus dem Lendenbereich. Sehr 
	mager.
	Ideal für elegante Gerichte, kurzgebraten oder gegrillt.
	
	\item [Country-Style Ribs]  
	Diese Rippchen stammen aus der Lende und enthalten oft mehr Fleisch als 
	Knochen.
	Hervorragend für langsames Garen und BBQ.
\end{description}

\newpage
\begin{figure}[htbp]
	\centering
	\begin{minipage}{1\textwidth}
		\centering
		\includegraphics[width=.9\linewidth]{pics/Übersicht_cut_Schwein}
		\captionof{figure}{Übersicht amerikanische Cuts vom Schwein}
		\label{fig:Übersicht1}
	\end{minipage}
\end{figure}
\newpage

\section{Spanische Zuschnitte vom Schwein}

\begin{description}

	\item [Chuletero] Der Kotelettstrang der auch als Karree bezeichnet wird. Er umfasst das Rippenstück
	beiderseits der Wirbelsäule hinter dem Nacken bis zur Hinterkeule. Daraus werden die Nacken-, Stiel- oder 
	und Filetkoteletts in Scheiben gewonnen. Diese werden in der Regel am Knochen zum bereitet. Sie werden 
    allerdings auch ausgelöst zubereitet, kann man machen, muss man aber nicht. 
	
	\item [Costillas] oder Kotelettrippchen sind ein richtiger Augenschmaus auf dem Grill. Dieser Zuschnitt für 
	Rippchen umfasst den kompletten Strang zwischen Wirbelsäule und Bauch. Ein feiner nussiger Geschmack 
	und ein ausgezeichneter, zarter Schmelz aufgrund des relativ hohen Fettanteils zeichnen diesen Cut aus.
	
	\item [Jamón Ibérico] aus den schwarzen Schweinen wird der Pata Negra Schinken hergestellt, der nicht mit 
	dem Jámon Serrano (der aus weißen Schweine hergestellt wird) zu verwechseln ist. Diese Delikatesse wird 
	nicht geräuchert sondern luftgetrocknet. Dieser Schinken ist sensationell, spielt aber beim BBQ nur in Form 
	von exzellenten Tapas eine Rolle. 
	
	\item [Lomo-Ibérico] Der Lachsrücken ist pariert und entvliest und weißt daher kein Fett, keine Sehnen oder 
	Häutchen auf. Der Iberico Rücken ist eines der hochwertigsten Stücke, ist relativ mager und sehr zart.
	
	\item [Papada] Schweinekinn, auch das ist bei uns wenig verbreitet. Bei diesem Cut handelt es sich 
	um einen Teil des dünnen Backenmuskels. Dieser spezielle Zuschnitt ist sehr saftig und ist angenehm bissfest.
	
	\item[Carrillera] Schweinbäckchen stammen aus der Unterkiefermuskulatur. Die kreuzweise verlaufenden Muskelfasern verleihen dem 
	Fleisch seine exklusive Textur. Das Fleisch ist perfekt geeignet für Schmorgerichte die im Dutch oven besonders gut gelingen. Für das 
	BBQ empfiehlt sich das indirekte Grillen.
	
	\item [Presa] Dieser Cut gilt insbesondere in Spanien als Delikatesse. Und 
	stammt aus dem Schweinenacken, genauer gesagt aus dem 
	Nackenkern. Das Teilstück hat eine besonders feine Marmorierung – 
	insbesondere dann, wenn es vom Iberico Schwein stammt. Presa überzeugt 
	als kurzgebratenes oder gegrilltes Steak oder auch als ganzer Braten.
	
	\item [Secreto] Dieses Stück war außerhalb Spaniens lange Zeit unbekannt 
	– daher auch der spanische Name, der ins deutsche übersetzt ">Geheimnis"< heißt. Auch jetzt gilt es noch 
	als Geheimtipp in unseren Breitengraden. Das Stück ist ein Muskel, der zwischen 
	Schweinerücken und Schulter liegt. Da man es regelrecht suchen muss, wird 
	es auch ">verstecktes Filet"< genannt. Dieser Cut hat ein sehr intensives 
	Aroma und eignet sich sowohl zum Kurzbraten als auch für ">slow and low"< 
	Kochmethoden.
	
	\item[Pluma] Auch Rückendeckel oder Federstück genannt, ist die dreieckige Spitze des Rückens vom 
	Schwein. Dieses hierzulande eher unbekannte flache Stück ist eines der delikatesten Teile vom 
	Iberico-Schwein. Aufgrund seiner starken Marmorierung bleibt dieses Teilstück herrlich saftig nach der 
	Zubereitung auf dem Grill oder in der Pfanne.
	
	\item[Solomillo] Das Lendenstück ist ein echtes Premiumfleisch. Es liegt unter Lende und wird aus dem Inneren des Lendenstücks 
	gewonnen. Das Solomillo ist ausgesprochen zart und buttrig in der Textur, da der Muskel kaum beansprucht wird. Die feine 
	Mamorierung in Kombination mit dem Oberflächengrill lassen ein subtiles Aroma entfalten.
\end{description}

Diese Fleischzuschnitte bieten eine Vielzahl von Zubereitungsmöglichkeiten 
und eignen sich für unterschiedliche Kochmethoden, von schnellem Anbraten 
bis hin zu langsamen Schmoren. Viel Spaß beim Entdecken der verschiedenen 
Schweinefleischzuschnitte und beim Zubereiten köstlicher Gerichte! 
\newpage
\begin{figure}[htbp]
	\centering
	\begin{minipage}{1\textwidth}
		\centering
		\includegraphics[width=.9\linewidth]{pics/Übersicht_cut_Iberico}
		\captionof{figure}{Übersicht spanischer Cuts vom Schwein}
		\label{fig:Übersicht3}
	\end{minipage}
\end{figure}
\newpage

\section{Fleischzuschnitte vom Lamm}
Hier ist eine umfassende Liste der gängigen Fleischzuschnitte vom Lamm siehe 
Abbildung~\vref{fig:Übersicht2}, zusammen mit einer kurzen Beschreibung 
ihrer Eigenschaften und Verwendungszwecke:

\begin{description}
	\item [Neck  (Nacken)]
	Durchs Grasen beanspruchen Lämmer ständig ihren Nacken. Der 
	Lammnacken ist dadurch kräftig im Geschmack und weist eine Marmorierung 
	mit feinem, aber festem Fett auf. Dadurch eignet sich das aromatische Stück 
	des Lamms nicht nur zum Anbraten in Steakform, sondern auch für 
	Low-and-Slow Verfahren und wird so beispielsweise zu Pulled Lamb 
	verarbeitet.

	\item [Shoulder (Schulter)]
	Da die Schulter beim Lamm mit einer recht speziellen Fettmarmorierung und 
	hohem Knochenanteil daherkommt, wird sie eher seltener genutzt. Doch das 
	ist dem Cut gegenüber unfair, denn das Fleisch ist typisch für Lamm 
	aromatisch wie geschmacksintensiv und kann variabel auf dem Grill landen. 
	Es eignet sich in Steakform zum Kurzbraten oder Grillen und kann als Braten 
	in Low-and-Slow Verfahren schmackhaft gegart werden.
	Es ist aber wichtig zu erwähnen, dass an diesem Cut unbedingt das Fett 
	großzügig entfernt wird, denn bei Lamm braucht das Fett weitaus länger zum 
	Garen als das Fleisch und kann so das Endprodukt zu stark beeinflussen. 
	Denn ist das Fleisch auf den Punkt, braucht das Fett noch Zeit zum Garen. 
	Ist das Fett fertig, dann ist das Fleisch schon mehr als durchgebraten.

	\item [Rippchen (Rib)]
	Auch beim Lamm eignen sich die Rippchen wunderbar zum Grillen. Die aus 
	der Brust getrennten Rippchen sind meist sieben bis acht Stück an der Zahl. 
	Anders als beispielsweise beim Schwein sind sie nicht sonderlich groß und 
	weisen auch keine Unmengen an Fleisch auf. Dennoch lohnt es sich die mit 
	einem angenehmen Anteil von Fleisch und Fett durchzogenen Rippchen des 
	Lamms zuzubereiten. Aufgrund dieser Beschaffenheit sollten sie aber in 
	langsamen Garmethoden zubereitet werden, zum Beispiel langsam auf dem 
	Grill indirekt gegart.
	
	\item [Rücken (Back)]
	Der Rücken des Lamms gilt neben dem Filet als einer der hochwertigeren 
	Cuts. Es wird, wie der Name es ja schon verrät, aus dem Rücken des Tiers 
	geschnitten und kann in diverse Cuts geschnitten werden. Geschmacklich ist 
	es dank des Fettrands intensiv, weist das typisch intensive Lammaroma auf 
	und kann mit zartem Fleisch überzeugen. Erwähnenswert ist beim 
	Lammrücken der Karree-Cut, der acht Rippenbögen des Lamms vereint und 
	dem erwähnten zarten wie kräftigen Geschmack aufwarten kann. Das Karree 
	bietet sich an am Stück gegart oder gegrillt zu werden. Ganz famos 
	angerichtet sind Karrees übrigens, wenn man mehrere zusammen als 
	Lammkrone aufgestellt serviert. Schneidet man das Karree zwischen den 
	Rippen in kleineren Portionen, erhält man Lammkoteletts, die sich wunderbar 
	eignen gebraten serviert zu werden.

	\item [Loin (Filet)]
	Wie bei allen anderen Nutztieren ist auch beim Lamm das Filet der qualitativ 
	beste Cut und wird am hinteren Teil des Rückens, dem Tenderloin, 
	ausgetrennt. Anders als bei anderen Tieren ist das Filet des Lamms 
	vergleichsweise klein. Ungeachtet der Größe weist das Filet beim Lamm aber 
	die für den Cut typischen Merkmale auf und ist ein mageres wie zartes Stück 
	Fleisch, aber überraschenderweise nicht das fettärmste Stück. Es eignet sich 
	aufgrund des Cuts optimal zum Kurzbraten oder Grillen.

	\item [Valentine Steak]
	Ein Cut, dessen Anblick und Geschmack zum Verlieben einlädt. In erster 
	Linie, weil der besondere Cut entfernt richtig zugeschnitten an die Form 
	eines Herz erinnert. Das Stück wird dabei aus dem Karree, also dem Rücken 
	des Lamms, in der Breite von zwei Koteletts herausgeschnitten. Dann wird 
	das Stück mittig in einem Schmetterlingsschnitt aufgeklappt und zum 
	optischen Highlight, sobald das überschüssige Fett getrimmt wurde. 
	Geschmacklich ist das Valentine Steak ein typisches Kotelett vom Lamm mit 
	herzhaften wie kräftigen Aroma, das sich zum Grillen anbietet.

	\item [Keule (Leg)]
	Der größte Cut des Lamms, den man kriegen kann, wird vom Hinterbein 
	geschnitten. Naja, besser gesagt, ist es meist das gesamte Hinterbein des 
	Lamms und weist ein sehr mageres, aber zugleich zartes und aromatisches 
	Fleisch auf. Die Zubereitungen für den Grill sind divers. Ob im Ganzen mit 
	oder ohne Knochen oder als Teilcuts wie Steaks oder gewürfelt auf Spießen: 
	die Möglichkeiten Lammkeule zuzubereiten sind vielseitig wie schmackhaft 
	und bringen das magerste Fleisch des Lamms perfekt zur Geltung!
\end{description}
\newpage
\begin{figure}[htbp]
	\centering
	\begin{minipage}{1\textwidth}
		\centering
		\includegraphics[width=.9\linewidth]{pics/Übersicht_cut_Lamm}
		\captionof{figure}{Übersicht Cuts vom Lamm}
		\label{fig:Übersicht2}
	\end{minipage}
\end{figure}
\newpage


